%\VignetteIndexEntry{sorvi}
%The above line is needed to remove a warning in R CMD check
\documentclass[a4paper,finnish]{article}
\usepackage[finnish]{babel}
\usepackage[utf8]{inputenc}
\selectlanguage{finnish}
\usepackage{amsmath,amsthm,amsfonts}
\usepackage{graphicx}
%\usepackage[T1]{fontenc}
%\usepackage[latin1]{inputenc}
%\usepackage[authoryear,round]{natbib}
\usepackage[numbers]{natbib}
\usepackage{hyperref}
\usepackage{Sweave}
\usepackage{float}

%\setlength{\textwidth}{417pt}
%\setlength{\oddsidemargin}{44pt}
%\setlength{\marginparwidth}{55pt}
\setlength{\parindent}{0mm}
\setlength{\parskip}{2mm}
%\setlength{\topmargin}{9pt}
%\addtolength{\textheight}{40pt}
%\textwidth=6.2in
%\textheight=8.5in
%\oddsidemargin=.1in
%\evensidemargin=.1in
%\headheight=-.3in

\newcommand{\Rfunction}[1]{{\texttt{#1}}}
\newcommand{\Robject}[1]{{\texttt{#1}}}
\newcommand{\Rpackage}[1]{{\textit{#1}}}

\title{soRvi\\avoimen datan ty{\"o}kalupakki}
\author{Leo Lahti\footnote{Wageningen University, Nederland <leo.lahti@iki.fi>}\ , Juuso Parkkinen\footnote{Aalto-yliopisto <juuso.parkkinen@gmail.com>}\ \ ja Joona Lehtom{\"a}ki\footnote{Helsingin yliopisto}}

\hyphenation{a-voi-men jul-kis-hal-lin-non tie-to-läh-tei-siin tar-jo-a-mal-la joh-da-tus e-si-merk-ke-jä pa-li-kan}

\begin{document}

\maketitle

\section{Johdanto}

Kaikille avointen yhteiskunnallisten tietovarantojen määrä on
voimakkaassa kasvussa \cite{Poikola10}. Talouteen, säähän,
paikkatietoon, liikenteeseen, koulutukseen ja muihin alueisiin
liittyvää dataa on alettu avaamaan julkishallinnon toimesta Suomessa
ja muualla. Aineistojen pöyhintä avaa uusia näkökulmia ja avoimuus
mahdollistaa yhdistämisen toisiin tietolähteisiin, jolloin voidaan
vastata kysymyksiin joihin yksittäisillä aineistoilla ei päästä
käsiksi. Laaja saatavuus ja käyttö voi siten merkittävästi nostaa
datan arvoa.

Laskennallisten työkalujen saatavuus on osoittautunut keskeiseksi
pullonkaulaksi. Tämä projekti pyrkii paikkaamaan puutetta tarjoamalla
yleiskäyttöisiä välineitä julkisten tietoaineistojen hakuun,
putsaamiseen, yhdistelyyn, louhintaan ja visualisointiin. Pakettiin on
koottu kattava kokoelma ratkaisuja avoimen datan käsittelyyn. Myös uudet
ehdotukset ja lisäykset ovat tervetulleita.

Yksityiskohtaisempia esimerkkejä paketin käytöstä Suomi-datan
penkomiseen löydät
Louhos-blogista\footnote{http://louhos.wordpress.com}. Ylläpitäjien
yhteystiedot löytyvät projektin kotisivulta
kautta\footnote{http://sorvi.r-forge.r-project.org}.


\section{Paketin asennus}

Yksityiskohtaiset asennusohjeet löytyvät osoitteesta\\
http://sorvi.r-forge.r-project.org/asennus.html

%\section{Esimerkkejä}
%Esimerkkejä paketin käytöstä löytyy blogista http://louhos.wordpress.com

\section{Pakettiin viittaaminen}

Paketin tarjoamat välineet ovat vapaasti käytettävissä
FreeBSD-lisenssillä\footnote{http://en.wikipedia.org/wiki/BSD\_licenses}. Mikäli
paketista on apua, toivomme viittaamista työhön \cite{sorvi11}. Lisää tietoa löytyy projektin kotisivulta\footnote{http://sorvi.r-forge.r-project.org/}.

\section{Julkiset tietokannat}

Alla esimerkkejä joidenkin julkisten aineistojen lataamisesta sorviin.

\subsection{Helsingin seudun ympäristöpalvelut}

Helsingin seudun
ympäristöpalveluiden\footnote{http://www.hsy.fi/seututieto/kaupunki/paikkatiedot/Sivut/Avoindata.aspx}
avoimia aineistoja ((C) HSY 2011) voi hakea valmiilla
hakurutiinilla. Seuraava esimerkki lataa väestötason tietoja ja kuvaa
ne Helsingin kartalla. Lisätietoja aineistoista löytyy HSYn
verkkosivuilta\footnote{http://www.hsy.fi/seututieto/Documents/Paikkatiedot/Tietokuvaukset\_kaikki.pdf}. 

\begin{Schunk}
\begin{Sinput}
> library(sorvi)
> sp <- GetHSY("Vaestoruudukko")
> df <- as.data.frame(sp) # investigate data contents
> at <-  c(seq(0, 2000, 250), Inf) 
> q <- PlotShape(sp, "ASUKKAITA", type = "oneway", at = at, ncol = length(at))
\end{Sinput}
\end{Schunk}

\subsection{Maanmittauslaitos}

Suomen karttatietoja on haettu Maanmittauslaitoksen
sivuilta\footnote{http://www.maanmittauslaitos.fi/aineistot-palvelut/digitaaliset-tuotteet/ilmaiset-aineistot/hankinta}
rajapintasyistä valmiiksi sorviin\footnote{Jatkokaytto sallittu - (C) MML 2011 - http://www.maanmittauslaitos.fi/node/6417}. Seuraava komento lataa valmiin listan MML:n shape-muotoisia aineistoja ja poimii kuntarajat 1:1000000 kartalta:

\begin{Schunk}
\begin{Sinput}
> data(MML)
> sp <- MML[["1_milj_Shape_etrs_shape"]][["kunta1_p"]]
\end{Sinput}
\end{Schunk}

% voi selata system.file-funktion avulla:
%<<MML2011A,results=hide,eval=FALSE>>=
%dir(system.file("extdata/Maanmittauslaitos/", package = "sorvi")) 
%@

Alkuperaisten shape-tiedostojen suora luku ja esikasittely onnistuu
seuraavasti:

\begin{Schunk}
\begin{Sinput}
> sp <- readShapePoly("kunta1_p.shp")
> sp <- PreprocessShapeMML(sp)      
\end{Sinput}
\end{Schunk}

\subsection{Tilastokeskus}

Tilastokeskuksen
sivuilta\footnote{http://www.stat.fi/tup/tilastotietokannat/index.html}
löytyy avoimia aineistoja PC-Axis muodossa. Tässä esimerkkinä poimittu
kuntien mediaanitulot (määrittele muuttujaan url-osoite
"http://pxweb2.stat.fi/Database/StatFin/tul/tvt/2009/120\_tvt\_2009\_2011-02-18\_tau\_112\_fi.px":

\begin{Schunk}
\begin{Sinput}
> library(pxR)
> px <- GetPXTilastokeskus(url)
\end{Sinput}
\end{Schunk}

\subsection{Suomen ympäristökeskus}

Suomen ympäristökeskuksen\footnote{http://www.ymparisto.fi/syke}
(SYKE) OIVA-palvelu\footnote{www.ymparisto.fi/oiva} sisältää lukuisia
avoimia aineistoja. Aineistot ovat saatavilla sekä latauspalvelun että
WMS-rajapinnan kautta. OIVA-palvelu vaati ilmaisen käyttäjätunnuksen,
WMS-palvelu ei edellytä tunnusta. Esimerkkejä näiden aineistojen
käytöstä löytyy Louhos-blogista.

\subsection{Muuta}

\subsubsection{Postinumerot}

\begin{Schunk}
\begin{Sinput}
> suomen.postinumerot <- GetPostalCodeInfo()
\end{Sinput}
\end{Schunk}


% Korvaa ja poista tama, jos tiedot loytyy jo
% Maanmittauksen datoista
%\subsubsection{Kunta-maakunta-lääni-taulukko}
%<<alueet, results = hide, eval = FALSE>>=
%alueet <- aluetaulukko.suomi()
%@

\subsubsection{Kuntatason informaatio}

Poimi kuntatason tietoja verkosta ja muodosta taulukko: 

\begin{Schunk}
\begin{Sinput}
> municipality.info <- GetMunicipalityInfo()
\end{Sinput}
\end{Schunk}

Etsi annetuille kunnille vastaava maakunta:

\begin{Schunk}
\begin{Sinput}
> kunnat <- as.character(municipality.info$Kunta)
> kunta.maakunta <- FindProvince(kunnat, municipality.info)
\end{Sinput}
\end{Schunk}

\subsubsection{Maakuntatason informaatio}

Poimi maakuntatason väestötiheystiedot Wikipediasta:

\begin{Schunk}
\begin{Sinput}
> url <- "http://fi.wikipedia.org/wiki/V%C3%A4est%C3%B6tiheys"
> vaestotiheys <- GetProvinceInfo(url)
\end{Sinput}
\end{Schunk}

\section{Versiotiedot}

\begin{Schunk}
\begin{Sinput}
> sessionInfo()
\end{Sinput}
\begin{Soutput}
R version 2.14.1 (2011-12-22)
Platform: x86_64-pc-linux-gnu (64-bit)

locale:
 [1] LC_CTYPE=en_GB.UTF-8       LC_NUMERIC=C              
 [3] LC_TIME=en_GB.UTF-8        LC_COLLATE=en_GB.UTF-8    
 [5] LC_MONETARY=en_GB.UTF-8    LC_MESSAGES=en_GB.UTF-8   
 [7] LC_PAPER=C                 LC_NAME=C                 
 [9] LC_ADDRESS=C               LC_TELEPHONE=C            
[11] LC_MEASUREMENT=en_GB.UTF-8 LC_IDENTIFICATION=C       

attached base packages:
[1] stats     graphics  grDevices utils     datasets  methods   base     

loaded via a namespace (and not attached):
[1] tools_2.14.1
\end{Soutput}
\end{Schunk}


%\bibliographystyle[numbers]{natbib} 

\begin{thebibliography}{1}

\bibitem{sorvi11}
Leo Lahti, Juuso Parkkinen ja Joona Lehtomäki (2011).
\newblock soRvi - avoimen datan työkalupaketti
\newblock URL: http://louhos.wordpress.com

\bibitem{Poikola10}
Antti Poikola, Petri Kola ja Kari A. Hintikka (2010).
\newblock Julkinen data – joh\-da\-tus tietovarantojen avaamiseen
\newblock Liikenne- ja Viestintäministeriö. Edita Prima Oy, Helsinki 2010.
\newblock URL: http://www.julkinendata.fi

%\bibliographystyle{abbrv}
%\bibliography{my.bib}
\end{thebibliography}

\end{document}
